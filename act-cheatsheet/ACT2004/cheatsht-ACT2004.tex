% Template pour faire aide-mémoire
\documentclass[10pt, french]{article}

%% -----------------------------
%% Préambule
%% -----------------------------
% !TEX encoding = UTF-8 Unicode
% LaTeX Preamble
% Author : Gabriel Crépeault-Cauchon

% HOW-TO : copy-paste this file in the same directory as your .tex file, and add in your preamble the next command right after you have specified your documentclass : 
% % !TEX encoding = UTF-8 Unicode
% LaTeX Preamble
% Author : Gabriel Crépeault-Cauchon
% Last update : 15/10/2018

% HOW-TO : copy-paste this file in the same directory as your .tex file, and add in your preamble the next command right after you have specified your documentclass : 
% % !TEX encoding = UTF-8 Unicode
% LaTeX Preamble
% Author : Gabriel Crépeault-Cauchon
% Last update : 15/10/2018

% HOW-TO : copy-paste this file in the same directory as your .tex file, and add in your preamble the next command right after you have specified your documentclass : 
% \input{preambule_utf8.tex}
% ---------------------------------------------
% ---------------------------------------------
%% BEGINNING OF PREAMBLE
% Encoding packages
\usepackage[utf8]{inputenc}
\usepackage[T1]{fontenc}
\usepackage{babel}
\usepackage{lmodern}

% HYPERREF (URL's and Link options)
\usepackage{hyperref}
\hypersetup{colorlinks = true, urlcolor = blue, linkcolor = black}

% POLICY (choose one of them)
%	\usepackage{concrete}
%	\usepackage{mathpazo}
%	\usepackage{frcursive} %% permet d'écrire en lettres attachées
%	\usepackage{aeguill}
	\usepackage{mathptmx}
%	\usepackage{fourier} 

% MATHEMATICS CONFIGURATION
\usepackage{amsmath,amsthm,amssymb,latexsym,amsfonts}
\usepackage{empheq}
\usepackage{numprint}


% TCOLORBOX CONFIGURATION
\usepackage{tcolorbox}
\tcbuselibrary{xparse}
\tcbuselibrary{breakable}
%% Définition Boite pour exemple
\newcounter{ex}[section]
\DeclareTColorBox{exemple}{ o }% #1 parameter
{colframe=green!20!black,colback=green!5!white, % color of the box
breakable, pad at break*=0mm, % to split the box
before title = {\textbf{Exemple \stepcounter{ex} \arabic{chapter}.\arabic{section}.\arabic{ex} }},
IfValueTF = {#1}{title= {#1}}{title= \hphantom},
after title = {\large \hfill \faWrench}
}% conditionnal usage : if a title is specified, use it, else put "Exemple"

%% Définition boite pour définition
\newcounter{def}[section]
\DeclareTColorBox{definition}{ o }% #1 parameter
{colframe=blue!60!green,colback=blue!5!white, % color of the box
breakable, pad at break*=0mm, % to split the box
before title = {\textbf{Définition \stepcounter{def} \arabic{chapter}.\arabic{section}.\arabic{def} }},
title = {#1},
after title = {\large \hfill \faBook}
}


% Graphics and picture import Packages
\usepackage{graphicx}
\usepackage{pict2e}

% insert PDF package
\usepackage{pdfpages}

% Color package
\usepackage{color, soulutf8, colortbl}

% usefull shortcut for colored text
\newcommand{\orange}{\textcolor{orange}}
\newcommand{\red}{\textcolor{red}}
\newcommand{\cyan}{\textcolor{cyan}}
\newcommand{\blue}{\textcolor{blue}}
\newcommand{\green}{\textcolor{green}}
\newcommand{\purple}{\textcolor{magenta}}
\newcommand{\yellow}{\textcolor{yellow}}

% Custum enumerate & itemize Package
\usepackage{enumitem}
% Change default label for itemize
\renewcommand{\labelitemi}{\faAngleRight}

% Mathematics shortcut macros
\newcommand{\reels}{\mathbb{R}}
\newcommand{\entiers}{\mathbb{Z}}
\newcommand{\naturels}{\mathbb{N}}
\newcommand{\eval}{\biggr \rvert}
\usepackage{cancel}
\newcommand{\derivee}[1]{\frac{\partial}{\partial #1}}
\newcommand{\prob}[1]{\Pr \left( #1 \right)}
\newcommand{\esp}[1]{\mathrm{E} \left[ #1 \right]}
\newcommand{\variance}[1]{\mathrm{VaR} \left( #1 \right)}
\newcommand{\laplace}{\mathcal{L}}
\newcommand{\indic}[1]{\mathds{1}_{\{ #1 \}}}

% Matricial notation (with simple command \bm{•})
\usepackage{bm}


% To indicate equation number on a specific line in align environment
\newcommand\numberthis{\addtocounter{equation}{1}\tag{\theequation}}

% Actuarial notation package
\usepackage{actuarialsymbol}
\usepackage{actuarialangle}

% Other shortcut
\newcommand{\p}{\paragraph{}}
\newcommand{\n}{\newline}

% Special symbols package
\usepackage[tikz]{bclogo}
\usepackage{fontawesome}

%% END OF PREAMBLE
% ---------------------------------------------
% ---------------------------------------------
% ---------------------------------------------
% ---------------------------------------------
%% BEGINNING OF PREAMBLE
% Encoding packages
\usepackage[utf8]{inputenc}
\usepackage[T1]{fontenc}
\usepackage{babel}
\usepackage{lmodern}

% HYPERREF (URL's and Link options)
\usepackage{hyperref}
\hypersetup{colorlinks = true, urlcolor = blue, linkcolor = black}

% POLICY (choose one of them)
%	\usepackage{concrete}
%	\usepackage{mathpazo}
%	\usepackage{frcursive} %% permet d'écrire en lettres attachées
%	\usepackage{aeguill}
	\usepackage{mathptmx}
%	\usepackage{fourier} 

% MATHEMATICS CONFIGURATION
\usepackage{amsmath,amsthm,amssymb,latexsym,amsfonts}
\usepackage{empheq}
\usepackage{numprint}


% TCOLORBOX CONFIGURATION
\usepackage{tcolorbox}
\tcbuselibrary{xparse}
\tcbuselibrary{breakable}
%% Définition Boite pour exemple
\newcounter{ex}[section]
\DeclareTColorBox{exemple}{ o }% #1 parameter
{colframe=green!20!black,colback=green!5!white, % color of the box
breakable, pad at break*=0mm, % to split the box
before title = {\textbf{Exemple \stepcounter{ex} \arabic{chapter}.\arabic{section}.\arabic{ex} }},
IfValueTF = {#1}{title= {#1}}{title= \hphantom},
after title = {\large \hfill \faWrench}
}% conditionnal usage : if a title is specified, use it, else put "Exemple"

%% Définition boite pour définition
\newcounter{def}[section]
\DeclareTColorBox{definition}{ o }% #1 parameter
{colframe=blue!60!green,colback=blue!5!white, % color of the box
breakable, pad at break*=0mm, % to split the box
before title = {\textbf{Définition \stepcounter{def} \arabic{chapter}.\arabic{section}.\arabic{def} }},
title = {#1},
after title = {\large \hfill \faBook}
}


% Graphics and picture import Packages
\usepackage{graphicx}
\usepackage{pict2e}

% insert PDF package
\usepackage{pdfpages}

% Color package
\usepackage{color, soulutf8, colortbl}

% usefull shortcut for colored text
\newcommand{\orange}{\textcolor{orange}}
\newcommand{\red}{\textcolor{red}}
\newcommand{\cyan}{\textcolor{cyan}}
\newcommand{\blue}{\textcolor{blue}}
\newcommand{\green}{\textcolor{green}}
\newcommand{\purple}{\textcolor{magenta}}
\newcommand{\yellow}{\textcolor{yellow}}

% Custum enumerate & itemize Package
\usepackage{enumitem}
% Change default label for itemize
\renewcommand{\labelitemi}{\faAngleRight}

% Mathematics shortcut macros
\newcommand{\reels}{\mathbb{R}}
\newcommand{\entiers}{\mathbb{Z}}
\newcommand{\naturels}{\mathbb{N}}
\newcommand{\eval}{\biggr \rvert}
\usepackage{cancel}
\newcommand{\derivee}[1]{\frac{\partial}{\partial #1}}
\newcommand{\prob}[1]{\Pr \left( #1 \right)}
\newcommand{\esp}[1]{\mathrm{E} \left[ #1 \right]}
\newcommand{\variance}[1]{\mathrm{VaR} \left( #1 \right)}
\newcommand{\laplace}{\mathcal{L}}
\newcommand{\indic}[1]{\mathds{1}_{\{ #1 \}}}

% Matricial notation (with simple command \bm{•})
\usepackage{bm}


% To indicate equation number on a specific line in align environment
\newcommand\numberthis{\addtocounter{equation}{1}\tag{\theequation}}

% Actuarial notation package
\usepackage{actuarialsymbol}
\usepackage{actuarialangle}

% Other shortcut
\newcommand{\p}{\paragraph{}}
\newcommand{\n}{\newline}

% Special symbols package
\usepackage[tikz]{bclogo}
\usepackage{fontawesome}

%% END OF PREAMBLE
% ---------------------------------------------
% ---------------------------------------------
% ---------------------------------------------
% ---------------------------------------------

%% -----------------------------
%% Encoding packages
%% -----------------------------
\usepackage[utf8]{inputenc}
\usepackage[T1]{fontenc}
\usepackage{babel}
\usepackage{lmodern}

%% -----------------------------
%% Variable definition
%% -----------------------------
\def\cours{Mathématiques actuarielles IARD1}
\def\sigle{ACT-2005}
\def\session{Automne 2018}
\def\auteur{Gabriel Crépeault-Cauchon}
\def\BackgroundColor{gray!20!white}
\def\SectionColor{red!80!white}
\def\SubSectionColor{red!30!black}


%% -----------------------------
%% Margin and layout
%% -----------------------------
\usepackage[hmargin=1cm, vmargin=1.7cm]{geometry}
\usepackage{multicol}

%% -----------------------------
%% URL and links
%% -----------------------------
\usepackage{hyperref}
\hypersetup{colorlinks = true, urlcolor = gray!60!black, linkcolor = black}

%% -----------------------------
%% Document policy (uncomment only one)
%% -----------------------------
%	\usepackage{concrete}
	\usepackage{mathpazo}
%	\usepackage{frcursive} %% permet d'écrire en lettres attachées
%	\usepackage{aeguill}
%	\usepackage{mathptmx}
%	\usepackage{fourier} 

%% -----------------------------
%% Math configuration
%% -----------------------------
\usepackage{amsmath,amsthm,amssymb,latexsym,amsfonts}
\usepackage{empheq}
\usepackage{numprint}

% Mathematics shortcut
\newcommand{\reels}{\mathbb{R}}
\newcommand{\entiers}{\mathbb{Z}}
\newcommand{\naturels}{\mathbb{N}}
\newcommand{\eval}{\biggr \rvert}
\usepackage{cancel}
\newcommand{\derivee}[1]{\frac{\partial}{\partial #1}}
\newcommand{\prob}[1]{\Pr \left( #1 \right)}
\newcommand{\esp}[1]{E \left[ #1 \right]}
\newcommand{\laplace}{\mathcal{L}}

% To indicate equation number on a specific line in align environment
\newcommand\numberthis{\addtocounter{equation}{1}\tag{\theequation}}

% Actuarial notation package
\usepackage{actuarialsymbol}
\usepackage{actuarialangle}




%% -----------------------------
%% tcolorbox configuration
%% -----------------------------
\usepackage{tcolorbox}
\tcbuselibrary{xparse}
\tcbuselibrary{breakable}

%% Définition boite pour définition
\DeclareTColorBox{definition}{ o }% #1 parameter
{colframe=blue!60!green,colback=blue!5!white, % color of the box
breakable, pad at break*=0mm, % to split the box
title = {#1},
after title = {\large \hfill \faBook}
}

%% -----------------------------
%% Graphics and pictures
%% -----------------------------
\usepackage{graphicx}
\usepackage{pict2e}

%% -----------------------------
%% insert pdf pages into document
%% -----------------------------
\usepackage{pdfpages}

%% -----------------------------
%% Color configuration
%% -----------------------------
\usepackage{color, soulutf8, colortbl}

% usefull shortcut for colored text
\newcommand{\orange}{\textcolor{orange}}
\newcommand{\red}{\textcolor{red}}
\newcommand{\cyan}{\textcolor{cyan}}
\newcommand{\blue}{\textcolor{blue}}
\newcommand{\green}{\textcolor{green}}
\newcommand{\purple}{\textcolor{magenta}}
\newcommand{\yellow}{\textcolor{yellow}}


%% -----------------------------
%% Enumerate environment configuration
%% -----------------------------
% Custum enumerate & itemize Package
\usepackage{enumitem}
% French Setup for itemize function
\frenchbsetup{StandardItemLabels=true}
% Change default label for itemize
\renewcommand{\labelitemi}{\faAngleRight}

%% -----------------------------
%% Tabular column type configuration
%% -----------------------------
\newcolumntype{C}{>{$}c<{$}} % math-mode version of "l" column type
\newcolumntype{L}{>{$}l<{$}} % math-mode version of "l" column type
\newcolumntype{R}{>{$}r<{$}} % math-mode version of "l" column type


%% -----------------------------
%% Fontawesome for special symbols
%% -----------------------------
\usepackage{fontawesome}

%% -----------------------------
%% Section Font customization
%% -----------------------------
\usepackage{sectsty}
\sectionfont{\color{\SectionColor}}
\subsectionfont{\color{\SubSectionColor}}

%% -----------------------------
%% Footer/Header Customization
%% -----------------------------


\usepackage{lastpage}
\usepackage{fancyhdr}
\pagestyle{fancy}
% Header
\fancyhead{} 	% Reset
\fancyhead[L]{Aide-mémoire pour~ \cours ~(\textbf{\sigle})}
\fancyhead[R]{\auteur}

% Footer
\fancyfoot{}		% Reset
\fancyfoot[R]{\thepage ~de~ \pageref{LastPage}}
\fancyfoot[L]{\href{https://github.com/gabrielcrepeault/latex-template}{\faGithub \ gabrielcrepeault/latex-template}}


\pagecolor{\BackgroundColor}




%% END OF PREAMBLE
% ---------------------------------------------
% ---------------------------------------------

%% -----------------------------
%% Début du document
%% -----------------------------

\usepackage{mathrsfs}
\begin{document}


% j'enlève le footnotesize temporairement, sinon je ne vois rien! GCC
% \footnotesize % Écrire petit (peut être modifié)
\begin{multicols*}{3} % Nombre de colonnes (peut être changé plus tard.)
\section*{Rappel de Math. financière}
\paragraph{Annuitées}
\begin{align*}
	\actsymb{a}{\angln} &= \frac{1 - v^t}{i} \\
	\actsymb{a}{\angl{\infty}} &= \frac{1}{i} \\
	\actsymb{\bar{a}}{\angln} &= \int_0^n v^t dt = \frac{1 - v^t}{\delta}
\end{align*}

\section{Matière examen 1}
\paragraph{Définitions de base}
\begin{itemize}
\item[$X$ : ] Âge au décès d'un nouveau-né
\item[$T_x$ : ] Durée de vie résiduelle d'un individu d'âge $x$.
\[T_x = (X-x | X \geq x) \]
\[ f_{T_x} = \px[t]{x} \mu_{x+t} \]
\[ F_{T_x}   =\qx[t]{x} = \frac{S_X(x) - S_X(x+t)}{S_X(x)}  \]
\[ \prob{t \leq T_x \leq t+u} = \qx[t|u]{x} = \px[x]{t} \qx[u]{x+t} \]
\[S_{T_x}(t) = \frac{S_x(x+t)}{S_X(x)} = \exp \left\{ - \int_{0}^{t} \mu_{x+s} ds \right\}  \]
\item[$K_x$ : ] Durée de vie résiduelle entière d'un individu d'âge $x$.
\[K_x = \lfloor T_x \rfloor \]
\[\prob{K_x = k} = \prob{\lfloor T_x \rfloor = k} = \px[k|]{x} \]

\item[$\mu_x$ : ] Force de mortalité pour $(x)$
\[\mu_x = \lim_{t \to 0} \frac{\qx[t]{x}}{t} = \derivee{x} \Big( \ln(S_X(x)\Big) \]
\[\mu_{x+t} = - \derivee{x} \Big( \ln ( \px[t]{x}) \Big) \]
\end{itemize}

\paragraph{Définitions des tables de mortalité}
\begin{itemize}
\item[$\ell_0$ : ] Nombre d'individus initial dans une cohorte.
\item[$\ell_x$ : ] Nombre d'invidu de la cohorte ayant survécu jusqu'à l'âge $x$.

\[\qx[t]{x} = \frac{\ell_x - \ell_{x+t}}{\ell_x}\]
\[\px[t]{x} = \frac{\ell_{x+t}}{\ell_x} \]
\[\qx[t|u]{x} = \frac{\dx[u]{x+t}}{\ell_x}\]

\item[$I_j(x)$ : ] Indicateur de survie du $j$\up{e} individu jusqu'à l'âge $x$.
\[I_j(x) \sim Bin(1, S_X(x))\]
\item[$\mathscr{L}_x$ : ] v.a. du nombre de survivants jusqu'à l'âge $x$.
\[\ell_x = \esp{\mathscr{L}_x} = \sum_{j=1}^{\ell_0} I_j(x)\]

\item[$\prescript{}{n}{\mathscr{D}}_x$ : ] v.a. du nombre de décès entre l'âge $x$ et $x+n$.
\[•\]
\[\Dx[x]{x} = \mathscr{L}_x - \mathscr{L}_{x+n}\]
\[\dx[n]{x} = \esp{\Dx[n]{x}} = \ell_{x} - \ell_{x+n} \]
\end{itemize}
\paragraph{Espérance de vie résiduelle}
\[\eringx{x} = \esp{T_x} =  \int_{0}^{\omega - x} t \px[t]{x} \mu_{x+t} dt= \int_{0}^{\omega - x} \px[t]{x} dt \]
\[\eringx{x:\angln} = \left( \int_{0}^{n} t \px[t]{x} \mu_{x+t} dt   \right) + n \cdot \px[n]{x} = \int_{0}^{n} \px[t]{x} dt \]


\paragraph{Hypothèses d'interpolation}
à terminer plus tard.

\paragraph{Loi de Moivre}
\begin{align*}
	X &\sim Uni(0, \omega) \\
	S_x(x) &= 1 - \frac{x}{\omega},\: 0 < x < \omega \\
	T_x &\sim Uni(0, \omega - x) \\
	S_{T_x}(t) &= 1 - \frac{t}{\omega - x},\: 0 < t < \omega - x
\end{align*}

\paragraph{Loi Exponentielle}
\begin{align*}
	x &\sim Exp(\mu) \\
	S_x(x) &= e^{-\mu x},\: x \geq 0 \\
	T_x &\sim Exp(\mu) \\
	S_{T_x}(t) &= e^{-\mu t},\: t \geq 0
\end{align*}

\section{Contrats d'assurance-vie}
Le paiement est soit en continu, soit à la fin de l'année ou à la fin de la $\frac{1}{m}$ d'année.

\paragraph{Assurance-vie entière} On verse le capital au décès de l'assuré

\begin{flalign*}
\Ax*{x} & = \int_{0}^{\omega - x} v^t \px[t]{x} \mu_{x+t} dt \\
\Ax{x}	& = \sum_{k=0}^{\omega - x - 1} v^{k+1} \qx[k|]{x} \\
	& = \sum_{k=0}^{\omega - x - 1} v^{k+1} \px[k]{x} \qx[]{x+k} \\
\end{flalign*}

\paragraph{Assurance-vie temporaire} On verse le capital au décès de l'assuré, s'il survient dans les $n$ prochaines années.
\begin{flalign*}
\Ax*{\termxn}	& = \int_{0}^{n} v^t \px[t]{x} \mu_{x+t} dt \\
\Ax{\termxn}		& = \sum_{k=0}^{n-1} v^{k+1} \qx[k|]{x} \\
	& = \sum_{k=0}^{n-1} v^{k+1} \px[k]{x} \qx[]{x+k} \\
\end{flalign*}

\paragraph{Assurance-vie dotation pure} On verse le capital à l'assuré si celui-ci est toujours en vie après $n$ années.
\begin{align*}
\Ax{\pureendowxn}	& = \px[n]{x} v^n = \Ex[m]{x} \\
\end{align*}
où $\Ex[m]{x}$ est un facteur d'actualisation actuarielle.

\paragraph{Assurance mixte} On verse le capital à l'assuré si il décède dans les $n$ prochaines années, ou si il est toujours en vie après cette période.
\begin{align*}
\Ax*{x:\angln}	& = \int_{0}^{n} v^t \px[t]{x} \mu_{x+t} dt + v^n \px[n]{x} \\
	& = \Ax*{\termxn} + \Ax{\pureendowxn} \\
\Ax{x:\angln}		& = \sum_{k=0}^{n-1} v^{k+1} \qx[k|]{x} + v^n \px[n]{x} \\
\end{align*}

\paragraph{Assurance différée} On verse le capital à l'assuré lors de son décès seulement si le décès survient dans plus de $m$ années \footnote{Interprétation : Une assurance-vie entière qui débute dans $m$ années.}

\begin{align*}
\Ax*[m|]{x}	& = \int_{m}^{\omega -x} v^t \px[t]{x} \mu_{x+t} dt \\
	& = v^m \px[m]{x} \int_{0}^{\omega - x - m} v^t \px[t]{x+m} \mu_{(x+m)+t} dt \\
	& = \Ex[m]{x} \Ax*{x+m} \\
\Ax[m|]{x}	& = \sum_{k=m}^{\omega - x -1} v^{k+1} \qx[k|]{x} \\
	& = \sum_{k=0}^{\omega - x - m - 1} v^{k+1+m} \qx[(k+m)|]{x} \\
	& = v^m \px[m]{x} \sum_{k=0}^{\omega - (x+m) - 1} v^{k+1} \px[k]{x+m} \qx[]{x+m+k} \\
	& = \Ex[m]{x} \Ax{x+m} \\
\end{align*}
\paragraph{Lien entre assurance différée, assurance vie entière et assurance-vie temporaire}

\begin{align*}
\Ax*[m|]{x}	& = \Ax*{x} - \Ax*{\termxn}
\end{align*}

% Je met quelques formules de increase qui permet une rapidité dans les calcules 
\paragraph{Assurance Vie entière croissante} On verse le capital au décès de l'assuré. Ce capital augmente chaque années.

\begin{align*}
	\IbA*_x &= \int_0^{\omega - x} t v^t \px[t]{x} \mu_{x+t} dt \\
		%&= \int_0^{\omega - x} \Ax*[s|]{x} ds \\ 
	\IA*_{x} &= \int_0^{\omega - x} (1 + \lfloor t \rfloor) v^t \px[t]{x} \mu_{x+t} dt \\
		&= \Ax*{x} + \Ax*[1|]{x} + \Ax*[2|]{x} + ...
\end{align*}

\paragraph{Assurance Vie temporaire croissante} On verse le capital au décès de l'assuré, s'il survient dans les $n$ prochaines années. Ce capital croit chaque années.

\begin{align*}
	\IbA*_{\termxn} &= \int_0^n t v^t \px[t]{x} \mu_{x+t} dt \\
	\IA*_{\termxn}  &= \int_0^n (1 + \lfloor t \rfloor) v^t \px[t]{x} \mu_{x+t} dt \\
		&= \actsymb{\bar{A}}{\nthtop{1}{x}:\angln} + \actsymb[1|]{\bar{A}}{\nthtop{1}{x}:\angl{n-1}} + ... + \actsymb[n-1|]{\bar{A}}{\nthtop{1}{x}:\angl1}
\end{align*}

\paragraph{Assurance vie entière croissante temporairement} On verse le capital au décès de l'assuré. Ce capital croit pendent $n$ années

\begin{align*}
	\twoletsymb{I_{\angln}}{\bar{A}}_{x}  &= \int_0^n (1 + \lfloor t \rfloor) v^t \px[t]{x} \mu_{x+t} dt \\
	& + \int_n^{\omega - x} n v^t \px[t]{x} \mu_{x+t} dt \\
		&= \Ax*{x} + \Ax*[1|]{x} + ... + \Ax*[n-1|]{x}
\end{align*}

\paragraph{Assurance Vie temporaire décroissante} On verse le capital au décès de l'assuré, s'il survient dans les $n$ prochaines années. Ce capital décroit chaque années.

\begin{align*}
	\DbA*_{\termxn} &= \int_0^{\omega - x} (n - t) v^t \px[t]{x} \mu_{x+t} dt \\
	\DA*_{\termxn}  &= \int_0^{\omega - x} (n - \lfloor t \rfloor) v^t \px[t]{x} \mu_{x+t} dt \\
		&= \actsymb{\bar{A}}{\nthtop{1}{x}:\angl1} + \actsymb{\bar{A}}{\nthtop{1}{x}:\angl2} + ... + \actsymb{\bar{A}}{\nthtop{1}{x}:\angln}
\end{align*}

\section{Contrats de rente}
\paragraph{Rente viagère} On verse une rente à l'assuré jusqu'à son décès. 
\begin{align*}
Y	&  = \ax*{\angl{T_{x}}} = \frac{1-v^{T_x}}{\delta} = \frac{1 -\overline{Z}_x}{\delta} \\
\ax*{x} & = \int_{0}^{\infty}  \ax*{\angl{T_{x}}}  \px[t]{x} \mu_{x+t} dt \\
	& = \int_{0}^{\infty} v^t \px[t]{x} dt \\
	& = \frac{1 - \Ax*{x}}{\delta} \\
Var(Y)	& = Var\left( \frac{1- v^{T_x}}{\delta} \right) =  \frac{\Ax*[][2]{x} - \Ax*{x}^2}{\delta^2} \\
\end{align*}

\paragraph{Rente temporaire $n$ années} Ce contrat de rentes prévoit payer une rente à l'assuré s'il est en vie, au maximum $n$ années.
\begin{align*}
Y & = \begin{cases}
\ax*{\angl{T_x}}	& , T_x < n \\
\ax*{\angln}			& , T_x \geq n \\
\end{cases} = \frac{1  - \overline{Z}_{x:\angln}}{\delta} \\
\ax*{x:\angln}	& = \int_{0}^{n} \ax*{\angl{t}} \  \px[t]{x} \  \mu_{x+t} dt \\
	& = \int_{0}^{n} v^t \px[t]{x} dt \\
	& = \frac{1 - \Ax*{x:\angln}}{\delta} \\
Var(Y)	& = \frac{\Ax*[][2]{x:\angln} - \Ax*{x:\angln}^2}{\delta^2} \\	
 \end{align*}

\paragraph{Rente viagère différée $m$ années} C'est un contrat de rente viagère, qui débute dans $m$ années (si $(x)$ est en vie).
\begin{align*}
Y & = \begin{cases}
0											& T_x	 < 	 m	\\
v^m \ax*{\angl{T_x - m}}	& T_x 	\geq m	\\
\end{cases} = \overline{Y}_x - \overline{Y}_{x:\angl{m}} \\
\ax*[m|]{x}	& = \int_{m}^{\infty} \ax*{t-m} \px[t]{x} \mu_{x+t} dt \\
	& = \Ex[m]{x} \ax*{x+m} \\
	& = \ax*{x} - \ax*{x:\angl{m}} \\
Var(Y)	& = 		
\end{align*}

\paragraph{Rente garantie (certaine) $n$ années} Le contrat prévoit une rente minimale de $n$ années, pouvant se prolonger jusqu'au décès de l'assuré.
\begin{align*}
Y & = \begin{cases}
\ax*{\angln}	& = T_x < n \\
\ax*{\angl{T_x}}	& = T_x \geq n \\
\end{cases} = \overline{Y}_{x:\angln} + \prescript{}{n|}{Y}_x \\
\ax*{\joint{x:\angln}}	& =\ax*{\angln} \cdot \qx[n]{x} + \int_{n}^{\infty} \ax*{\angl{t}} \px[t]{x} \mu_{x+t} dt \\
	& = \ax*{\angln} + \ax*[n|]{x} \\
\end{align*}


\end{multicols*}
%% -----------------------------
%% Fin du document
%% -----------------------------
\end{document}