% !TEX encoding = UTF-8 Unicode
% LaTeX Preamble
% Author : Gabriel Crépeault-Cauchon

% HOW-TO : copy-paste this file in the same directory as your .tex file, and add in your preamble the next command right after you have specified your documentclass : 
% \input{preamble-cheatsht.tex}
% ---------------------------------------------
% ---------------------------------------------

%% -----------------------------
%% Encoding packages
%% -----------------------------
\usepackage[utf8]{inputenc}
\usepackage[T1]{fontenc}
\usepackage{babel}
\usepackage{lmodern}

%% -----------------------------
%% Variable definition
%% -----------------------------


\def\cours{Mathématiques actuarielles Vie 1}
\def\sigle{ACT-2004}
\def\session{Automne 2018}
\def\auteur{Gabriel Crépeault-Cauchon // Nicholas Langevin}
\def\BackgroundColor{white}
\def\SectionColor{burntorange}
\def\SubSectionColor{burntsienna}


%% -----------------------------
%% Margin and layout
%% -----------------------------
% Determine the margin for cheatsheet
\usepackage[landscape, hmargin=1cm, vmargin=1.7cm]{geometry}
\usepackage{multicol}

% Remove automatic indentation after section/subsection title.
\setlength{\parindent}{0cm}

% Save space in cheatsheet by removing space between align environment and normal text.
\usepackage{etoolbox}
\newcommand{\zerodisplayskips}{%
  \setlength{\abovedisplayskip}{0pt}%
  \setlength{\belowdisplayskip}{0pt}%
  \setlength{\abovedisplayshortskip}{0pt}%
  \setlength{\belowdisplayshortskip}{0pt}}
\appto{\normalsize}{\zerodisplayskips}
\appto{\small}{\zerodisplayskips}
\appto{\footnotesize}{\zerodisplayskips}

%% -----------------------------
%% URL and links
%% -----------------------------
\usepackage{hyperref}
\hypersetup{colorlinks = true, urlcolor = gray!70!white, linkcolor = black}

%% -----------------------------
%% Document policy (uncomment only one)
%% -----------------------------
%	\usepackage{concrete}
	\usepackage{mathpazo}
%	\usepackage{frcursive} %% permet d'écrire en lettres attachées
%	\usepackage{aeguill}
%	\usepackage{mathptmx}
%	\usepackage{fourier} 

%% -----------------------------
%% Math configuration
%% -----------------------------
\usepackage[fleqn]{amsmath}
\usepackage{amsthm,amssymb,latexsym,amsfonts}
\usepackage{empheq}
\usepackage{numprint}

% Mathematics shortcut
\newcommand{\reels}{\mathbb{R}}
\newcommand{\entiers}{\mathbb{Z}}
\newcommand{\naturels}{\mathbb{N}}
\newcommand{\eval}{\biggr \rvert}
\usepackage{cancel}
\newcommand{\derivee}[1]{\frac{\partial}{\partial #1}}
\newcommand{\prob}[1]{\Pr \left( #1 \right)}
\newcommand{\esp}[1]{E \left[ #1 \right]}
\newcommand{\laplace}{\mathcal{L}}

% To indicate equation number on a specific line in align environment
\newcommand\numberthis{\addtocounter{equation}{1}\tag{\theequation}}

% Actuarial notation package
\usepackage{actuarialsymbol}
\usepackage{actuarialangle}

% Matricial anotation for math symbols (\bm{•})
\usepackage{bm}
% matricial notation variable (bold style)
\newcommand{\matr}[1]{\mathbf{#1}}



%% -----------------------------
%% tcolorbox configuration
%% -----------------------------
\usepackage{tcolorbox}
\tcbuselibrary{xparse}
\tcbuselibrary{breakable}

%% Définition boite pour définition
\DeclareTColorBox{definition}{ o }% #1 parameter
{colframe=blue!60!green,colback=blue!5!white, % color of the box
breakable, pad at break*=0mm, % to split the box
title = {#1},
after title = {\large \hfill \faBook}
}

%% -----------------------------
%% Graphics and pictures
%% -----------------------------
\usepackage{graphicx}
\usepackage{pict2e}

%% -----------------------------
%% insert pdf pages into document
%% -----------------------------
\usepackage{pdfpages}

%% -----------------------------
%% Color configuration
%% -----------------------------
\usepackage{color, soulutf8, colortbl}


% New color definition
% Source : http://latexcolor.com
\definecolor{burntorange}{rgb}{0.8, 0.33, 0.0}
\definecolor{burntsienna}{rgb}{0.91, 0.45, 0.32}

% usefull shortcut for colored text
\newcommand{\orange}{\textcolor{orange}}
\newcommand{\red}{\textcolor{red}}
\newcommand{\cyan}{\textcolor{cyan}}
\newcommand{\blue}{\textcolor{blue}}
\newcommand{\green}{\textcolor{green}}
\newcommand{\purple}{\textcolor{magenta}}
\newcommand{\yellow}{\textcolor{yellow}}


%% -----------------------------
%% Enumerate environment configuration
%% -----------------------------
% Custum enumerate & itemize Package
\usepackage{enumitem}
% French Setup for itemize function
\frenchbsetup{StandardItemLabels=true}
% Change default label for itemize
\renewcommand{\labelitemi}{\faAngleRight}

%% -----------------------------
%% Tabular column type configuration
%% -----------------------------
\newcolumntype{C}{>{$}c<{$}} % math-mode version of "l" column type
\newcolumntype{L}{>{$}l<{$}} % math-mode version of "l" column type
\newcolumntype{R}{>{$}r<{$}} % math-mode version of "l" column type
\newcolumntype{f}{>{\columncolor{green!20!white}}p{1cm}}
% configuration to force a line break within a single cell
\usepackage{makecell}



%% -----------------------------
%% Fontawesome for special symbols
%% -----------------------------
\usepackage{fontawesome}

%% -----------------------------
%% Section Font customization
%% -----------------------------
\usepackage{sectsty}
\sectionfont{\color{\SectionColor}}
\subsectionfont{\color{\SubSectionColor}}

%% -----------------------------
%% Footer/Header Customization
%% -----------------------------
\usepackage{lastpage}
\usepackage{fancyhdr}
\pagestyle{fancy}
% Header
\fancyhead{} 	% Reset
\fancyhead[L]{Aide-mémoire pour~ \cours ~(\textbf{\sigle})}
\fancyhead[R]{\auteur}

% Footer
\fancyfoot{}		% Reset
\fancyfoot[R]{\thepage ~de~ \pageref{LastPage}}
\fancyfoot[L]{\href{https://github.com/gabrielcrepeault/latex-template}{\faGithub \ gabrielcrepeault/latex-template}}

% page background color
\pagecolor{\BackgroundColor}






%% END OF PREAMBLE
% ---------------------------------------------
% ---------------------------------------------