\documentclass[10pt, french]{article}

%% -----------------------------
%% Préambule
%% -----------------------------
% !TEX encoding = UTF-8 Unicode
% LaTeX Preamble
% Author : Gabriel Crépeault-Cauchon

% HOW-TO : copy-paste this file in the same directory as your .tex file, and add in your preamble the next command right after you have specified your documentclass : 
% % !TEX encoding = UTF-8 Unicode
% LaTeX Preamble
% Author : Gabriel Crépeault-Cauchon
% Last update : 15/10/2018

% HOW-TO : copy-paste this file in the same directory as your .tex file, and add in your preamble the next command right after you have specified your documentclass : 
% % !TEX encoding = UTF-8 Unicode
% LaTeX Preamble
% Author : Gabriel Crépeault-Cauchon
% Last update : 15/10/2018

% HOW-TO : copy-paste this file in the same directory as your .tex file, and add in your preamble the next command right after you have specified your documentclass : 
% \input{preambule_utf8.tex}
% ---------------------------------------------
% ---------------------------------------------
%% BEGINNING OF PREAMBLE
% Encoding packages
\usepackage[utf8]{inputenc}
\usepackage[T1]{fontenc}
\usepackage{babel}
\usepackage{lmodern}

% HYPERREF (URL's and Link options)
\usepackage{hyperref}
\hypersetup{colorlinks = true, urlcolor = blue, linkcolor = black}

% POLICY (choose one of them)
%	\usepackage{concrete}
%	\usepackage{mathpazo}
%	\usepackage{frcursive} %% permet d'écrire en lettres attachées
%	\usepackage{aeguill}
	\usepackage{mathptmx}
%	\usepackage{fourier} 

% MATHEMATICS CONFIGURATION
\usepackage{amsmath,amsthm,amssymb,latexsym,amsfonts}
\usepackage{empheq}
\usepackage{numprint}


% TCOLORBOX CONFIGURATION
\usepackage{tcolorbox}
\tcbuselibrary{xparse}
\tcbuselibrary{breakable}
%% Définition Boite pour exemple
\newcounter{ex}[section]
\DeclareTColorBox{exemple}{ o }% #1 parameter
{colframe=green!20!black,colback=green!5!white, % color of the box
breakable, pad at break*=0mm, % to split the box
before title = {\textbf{Exemple \stepcounter{ex} \arabic{chapter}.\arabic{section}.\arabic{ex} }},
IfValueTF = {#1}{title= {#1}}{title= \hphantom},
after title = {\large \hfill \faWrench}
}% conditionnal usage : if a title is specified, use it, else put "Exemple"

%% Définition boite pour définition
\newcounter{def}[section]
\DeclareTColorBox{definition}{ o }% #1 parameter
{colframe=blue!60!green,colback=blue!5!white, % color of the box
breakable, pad at break*=0mm, % to split the box
before title = {\textbf{Définition \stepcounter{def} \arabic{chapter}.\arabic{section}.\arabic{def} }},
title = {#1},
after title = {\large \hfill \faBook}
}


% Graphics and picture import Packages
\usepackage{graphicx}
\usepackage{pict2e}

% insert PDF package
\usepackage{pdfpages}

% Color package
\usepackage{color, soulutf8, colortbl}

% usefull shortcut for colored text
\newcommand{\orange}{\textcolor{orange}}
\newcommand{\red}{\textcolor{red}}
\newcommand{\cyan}{\textcolor{cyan}}
\newcommand{\blue}{\textcolor{blue}}
\newcommand{\green}{\textcolor{green}}
\newcommand{\purple}{\textcolor{magenta}}
\newcommand{\yellow}{\textcolor{yellow}}

% Custum enumerate & itemize Package
\usepackage{enumitem}
% Change default label for itemize
\renewcommand{\labelitemi}{\faAngleRight}

% Mathematics shortcut macros
\newcommand{\reels}{\mathbb{R}}
\newcommand{\entiers}{\mathbb{Z}}
\newcommand{\naturels}{\mathbb{N}}
\newcommand{\eval}{\biggr \rvert}
\usepackage{cancel}
\newcommand{\derivee}[1]{\frac{\partial}{\partial #1}}
\newcommand{\prob}[1]{\Pr \left( #1 \right)}
\newcommand{\esp}[1]{\mathrm{E} \left[ #1 \right]}
\newcommand{\variance}[1]{\mathrm{VaR} \left( #1 \right)}
\newcommand{\laplace}{\mathcal{L}}
\newcommand{\indic}[1]{\mathds{1}_{\{ #1 \}}}

% Matricial notation (with simple command \bm{•})
\usepackage{bm}


% To indicate equation number on a specific line in align environment
\newcommand\numberthis{\addtocounter{equation}{1}\tag{\theequation}}

% Actuarial notation package
\usepackage{actuarialsymbol}
\usepackage{actuarialangle}

% Other shortcut
\newcommand{\p}{\paragraph{}}
\newcommand{\n}{\newline}

% Special symbols package
\usepackage[tikz]{bclogo}
\usepackage{fontawesome}

%% END OF PREAMBLE
% ---------------------------------------------
% ---------------------------------------------
% ---------------------------------------------
% ---------------------------------------------
%% BEGINNING OF PREAMBLE
% Encoding packages
\usepackage[utf8]{inputenc}
\usepackage[T1]{fontenc}
\usepackage{babel}
\usepackage{lmodern}

% HYPERREF (URL's and Link options)
\usepackage{hyperref}
\hypersetup{colorlinks = true, urlcolor = blue, linkcolor = black}

% POLICY (choose one of them)
%	\usepackage{concrete}
%	\usepackage{mathpazo}
%	\usepackage{frcursive} %% permet d'écrire en lettres attachées
%	\usepackage{aeguill}
	\usepackage{mathptmx}
%	\usepackage{fourier} 

% MATHEMATICS CONFIGURATION
\usepackage{amsmath,amsthm,amssymb,latexsym,amsfonts}
\usepackage{empheq}
\usepackage{numprint}


% TCOLORBOX CONFIGURATION
\usepackage{tcolorbox}
\tcbuselibrary{xparse}
\tcbuselibrary{breakable}
%% Définition Boite pour exemple
\newcounter{ex}[section]
\DeclareTColorBox{exemple}{ o }% #1 parameter
{colframe=green!20!black,colback=green!5!white, % color of the box
breakable, pad at break*=0mm, % to split the box
before title = {\textbf{Exemple \stepcounter{ex} \arabic{chapter}.\arabic{section}.\arabic{ex} }},
IfValueTF = {#1}{title= {#1}}{title= \hphantom},
after title = {\large \hfill \faWrench}
}% conditionnal usage : if a title is specified, use it, else put "Exemple"

%% Définition boite pour définition
\newcounter{def}[section]
\DeclareTColorBox{definition}{ o }% #1 parameter
{colframe=blue!60!green,colback=blue!5!white, % color of the box
breakable, pad at break*=0mm, % to split the box
before title = {\textbf{Définition \stepcounter{def} \arabic{chapter}.\arabic{section}.\arabic{def} }},
title = {#1},
after title = {\large \hfill \faBook}
}


% Graphics and picture import Packages
\usepackage{graphicx}
\usepackage{pict2e}

% insert PDF package
\usepackage{pdfpages}

% Color package
\usepackage{color, soulutf8, colortbl}

% usefull shortcut for colored text
\newcommand{\orange}{\textcolor{orange}}
\newcommand{\red}{\textcolor{red}}
\newcommand{\cyan}{\textcolor{cyan}}
\newcommand{\blue}{\textcolor{blue}}
\newcommand{\green}{\textcolor{green}}
\newcommand{\purple}{\textcolor{magenta}}
\newcommand{\yellow}{\textcolor{yellow}}

% Custum enumerate & itemize Package
\usepackage{enumitem}
% Change default label for itemize
\renewcommand{\labelitemi}{\faAngleRight}

% Mathematics shortcut macros
\newcommand{\reels}{\mathbb{R}}
\newcommand{\entiers}{\mathbb{Z}}
\newcommand{\naturels}{\mathbb{N}}
\newcommand{\eval}{\biggr \rvert}
\usepackage{cancel}
\newcommand{\derivee}[1]{\frac{\partial}{\partial #1}}
\newcommand{\prob}[1]{\Pr \left( #1 \right)}
\newcommand{\esp}[1]{\mathrm{E} \left[ #1 \right]}
\newcommand{\variance}[1]{\mathrm{VaR} \left( #1 \right)}
\newcommand{\laplace}{\mathcal{L}}
\newcommand{\indic}[1]{\mathds{1}_{\{ #1 \}}}

% Matricial notation (with simple command \bm{•})
\usepackage{bm}


% To indicate equation number on a specific line in align environment
\newcommand\numberthis{\addtocounter{equation}{1}\tag{\theequation}}

% Actuarial notation package
\usepackage{actuarialsymbol}
\usepackage{actuarialangle}

% Other shortcut
\newcommand{\p}{\paragraph{}}
\newcommand{\n}{\newline}

% Special symbols package
\usepackage[tikz]{bclogo}
\usepackage{fontawesome}

%% END OF PREAMBLE
% ---------------------------------------------
% ---------------------------------------------
% ---------------------------------------------
% ---------------------------------------------

%% -----------------------------
%% Encoding packages
%% -----------------------------
\usepackage[utf8]{inputenc}
\usepackage[T1]{fontenc}
\usepackage{babel}
\usepackage{lmodern}

%% -----------------------------
%% Variable definition
%% -----------------------------
\def\cours{Mathématiques actuarielles IARD1}
\def\sigle{ACT-2005}
\def\session{Automne 2018}
\def\auteur{Gabriel Crépeault-Cauchon}
\def\BackgroundColor{gray!20!white}
\def\SectionColor{red!80!white}
\def\SubSectionColor{red!30!black}


%% -----------------------------
%% Margin and layout
%% -----------------------------
\usepackage[hmargin=1cm, vmargin=1.7cm]{geometry}
\usepackage{multicol}

%% -----------------------------
%% URL and links
%% -----------------------------
\usepackage{hyperref}
\hypersetup{colorlinks = true, urlcolor = gray!60!black, linkcolor = black}

%% -----------------------------
%% Document policy (uncomment only one)
%% -----------------------------
%	\usepackage{concrete}
	\usepackage{mathpazo}
%	\usepackage{frcursive} %% permet d'écrire en lettres attachées
%	\usepackage{aeguill}
%	\usepackage{mathptmx}
%	\usepackage{fourier} 

%% -----------------------------
%% Math configuration
%% -----------------------------
\usepackage{amsmath,amsthm,amssymb,latexsym,amsfonts}
\usepackage{empheq}
\usepackage{numprint}

% Mathematics shortcut
\newcommand{\reels}{\mathbb{R}}
\newcommand{\entiers}{\mathbb{Z}}
\newcommand{\naturels}{\mathbb{N}}
\newcommand{\eval}{\biggr \rvert}
\usepackage{cancel}
\newcommand{\derivee}[1]{\frac{\partial}{\partial #1}}
\newcommand{\prob}[1]{\Pr \left( #1 \right)}
\newcommand{\esp}[1]{E \left[ #1 \right]}
\newcommand{\laplace}{\mathcal{L}}

% To indicate equation number on a specific line in align environment
\newcommand\numberthis{\addtocounter{equation}{1}\tag{\theequation}}

% Actuarial notation package
\usepackage{actuarialsymbol}
\usepackage{actuarialangle}




%% -----------------------------
%% tcolorbox configuration
%% -----------------------------
\usepackage{tcolorbox}
\tcbuselibrary{xparse}
\tcbuselibrary{breakable}

%% Définition boite pour définition
\DeclareTColorBox{definition}{ o }% #1 parameter
{colframe=blue!60!green,colback=blue!5!white, % color of the box
breakable, pad at break*=0mm, % to split the box
title = {#1},
after title = {\large \hfill \faBook}
}

%% -----------------------------
%% Graphics and pictures
%% -----------------------------
\usepackage{graphicx}
\usepackage{pict2e}

%% -----------------------------
%% insert pdf pages into document
%% -----------------------------
\usepackage{pdfpages}

%% -----------------------------
%% Color configuration
%% -----------------------------
\usepackage{color, soulutf8, colortbl}

% usefull shortcut for colored text
\newcommand{\orange}{\textcolor{orange}}
\newcommand{\red}{\textcolor{red}}
\newcommand{\cyan}{\textcolor{cyan}}
\newcommand{\blue}{\textcolor{blue}}
\newcommand{\green}{\textcolor{green}}
\newcommand{\purple}{\textcolor{magenta}}
\newcommand{\yellow}{\textcolor{yellow}}


%% -----------------------------
%% Enumerate environment configuration
%% -----------------------------
% Custum enumerate & itemize Package
\usepackage{enumitem}
% French Setup for itemize function
\frenchbsetup{StandardItemLabels=true}
% Change default label for itemize
\renewcommand{\labelitemi}{\faAngleRight}

%% -----------------------------
%% Tabular column type configuration
%% -----------------------------
\newcolumntype{C}{>{$}c<{$}} % math-mode version of "l" column type
\newcolumntype{L}{>{$}l<{$}} % math-mode version of "l" column type
\newcolumntype{R}{>{$}r<{$}} % math-mode version of "l" column type


%% -----------------------------
%% Fontawesome for special symbols
%% -----------------------------
\usepackage{fontawesome}

%% -----------------------------
%% Section Font customization
%% -----------------------------
\usepackage{sectsty}
\sectionfont{\color{\SectionColor}}
\subsectionfont{\color{\SubSectionColor}}

%% -----------------------------
%% Footer/Header Customization
%% -----------------------------


\usepackage{lastpage}
\usepackage{fancyhdr}
\pagestyle{fancy}
% Header
\fancyhead{} 	% Reset
\fancyhead[L]{Aide-mémoire pour~ \cours ~(\textbf{\sigle})}
\fancyhead[R]{\auteur}

% Footer
\fancyfoot{}		% Reset
\fancyfoot[R]{\thepage ~de~ \pageref{LastPage}}
\fancyfoot[L]{\href{https://github.com/gabrielcrepeault/latex-template}{\faGithub \ gabrielcrepeault/latex-template}}


\pagecolor{\BackgroundColor}




%% END OF PREAMBLE
% ---------------------------------------------
% ---------------------------------------------


%% -----------------------------
%% Début du document
%% -----------------------------
\begin{document}



\small
\begin{multicols*}{3} % Nombre de colonnes (peut être changé plus tard.)
\section{Régression linéaire simple}
\subsection*{Postulats}
\begin{enumerate}[label=$\mathbf{H}_{\arabic*}$]
\item Linéarité : $\esp{\varepsilon_i} = 0$
\item Homoscédasticité : $Var(\varepsilon_i)= \sigma^2$
\item Indépendance : $Cov(\varepsilon_i, \varepsilon_j) = 0$
\item Normalité : $\varepsilon_i \sim N(0, \sigma^2)$
\end{enumerate}
\subsection*{Modèle}
\begin{align*}
\esp{Y_i | x_i} 	& = \beta_0 + \beta_1 x_i \\
Var(Y_i | x_i)	& = \sigma^2 \\
Y_i | x_i & \overset{\mathbf{H}_4}{\sim} N(\beta_0 + \beta_1 x_i, \sigma^2) \\
\end{align*}

\subsection*{Estimation des paramètres}
\begin{align*}
\hat{\beta}_0 	& = \bar{Y} - \hat{\beta}_1 \bar{x} \\
\hat{\beta}_1	& = \frac{\sum_{i=1}^{n} x_i Y_i - \bar{Y} \sum_{i=1}^{n} x_i}{\sum_{i=1}^{n} x_i^2 - \bar{x} \sum_{i=1}^{n} x_i} = \frac{S_{XY}}{S_{XX}}
\end{align*}

\subsection*{Estimation de $\sigma^2$}
\begin{align*}
\hat{\sigma^2} = s^2 = \frac{\sum_{i=1}^{n} \hat{\varepsilon_i}^2}{n-p'} = \frac{\sum_{i=1}^{n} (Y_i - \hat{Y}_i)^2}{n-2}
\end{align*}

\subsection*{Propriété des estimateurs}
\begin{align*}
\esp{\hat{\beta}_1} = \beta_1 \quad , Var(\hat{\beta}_1) = \frac{\sigma^2}{S_{XX}} \\
\hat{\beta}_1 \overset{H_4}{\sim} N(\beta_1, \frac{\sigma^2}{S_{XX}}) \\
\esp{\hat{\beta}_0} = \beta_0 \quad , Var(\hat{\beta}_0) = \sigma^2 \left( \frac{1}{n} + \frac{\bar{X}^2}{S_{XX}} \right) \\
\hat{\beta}_0 \overset{H_4}{\sim} N(\beta_0, \sigma^2 \left( \frac{1}{n}  + \frac{\bar{x}^2}{S_{XX}} \right) \\
Cov(\hat{\beta}_0, \hat{\beta}_1) = - \frac{\bar{x} \sigma^2}{S_{XX}}
\end{align*}

\subsection*{Tests d'hypothèse sur les paramètres}
$H_0$ : $\hat{\beta} = \theta_0$ , $H_1$ : $\hat{\beta} \neq \theta_0$
\begin{align*}
t_{obs} = \frac{\hat{\beta} - \theta_0}{\sqrt{\hat{Var(\hat{\beta}}}} \sim T_{n-2}
\end{align*}
On Rejette $H_0$ si $t_{obs} > | t_{n-2} (1 - \frac{\alpha}{2})|$

\subsection*{Intervalle de confiance}
\subsubsection*{Pour la droite de régression ($\esp{Y_0|x_0}$)}
Sachant que $\esp{Y_0 | x_0} = \beta_0 + \beta_1 x_0$, on a l'IC suivant
\begin{align*}
\hat{Y_0} \pm t_{n-2} \left(\frac{\alpha}{2} \right) \sqrt{s^2 \left( \frac{1}{n} + \frac{(x_0 - \bar{x})^2}{S_{XX}} \right)}
\end{align*}

\subsubsection*{Pour la prévision de $Y_0$}
Sachant que $Y_0 = \beta_0 + \beta_1 x_0 + \varepsilon$, on a l'IC suivant
\begin{align*}
\hat{Y_0} \pm t_{n-2} \left(\frac{\alpha}{2} \right) \sqrt{s^2 \left( 1 + \frac{1}{m} + \frac{(x_0 - \bar{x})^2}{S_{XX}} \right)}
\end{align*}

\subsection*{Analyse de la variance (ANOVA)}
\begin{tabular}{|f| *{4}{C|}}
\hline
\rowcolor{green!40!black} Source & \text{dl} & \text{SS} & \text{MS} & F \\\hline
Model & p & \thead{\sum_{i=1}^{n} (\hat{Y}_i - \bar{Y})^2 \\ (SSR)}  & \thead{SSR / dl_1 \\ (MSR)} & \frac{MSR}{MSE} \\\hline
Residual error & n - p' & \thead{\sum_{i=1}^{n} (Y_i - \hat{Y}_i)^2 \\ (SSE)} & \thead{SSE / dl_2 \\(MSE = s^2)} & \cellcolor{gray!30!white} \\\hline
Total & n - 1 & \thead{\sum_{i=1}^{n} (Y_i - \bar{Y})^2 \\ (SST)} & \cellcolor{gray!30!white} & \cellcolor{gray!30!white} \\\hline
\end{tabular}

\subsubsection*{Coefficient de détermination}
\begin{align*}
R^2 = \frac{SSR}{SST} = 1 - \frac{SSE}{SST}
\end{align*}
On a aussi la relation suivante avec $F_{obs}$ : 
\begin{align*}
F = \frac{R^2}{1 - R^2} \cdot \frac{n-p'}{p}
\end{align*}

\subsection*{Test F de Fisher pour la validité globale de la régression}
On rejette $H_0 : \beta_1 =  \beta_2 = ... =  \beta_p = 0$ si 
\begin{align*}
F_{obs} = \frac{MSR}{MSE} \geq F_{n, n-p'}(1 - \alpha)
\end{align*}
où $p$ est le nombre de variables explicatives dans le modèle (régression linéaire simple, $p=1$ et $p' = p+1$.



\subsection*{Distribution d'un résidu $\varepsilon$}
\begin{align*}
\esp{\hat{\varepsilon_i}} = 0 \quad , Var \left( \hat{\varepsilon_i} \right) = \sigma^2 (1 - h_{ii})
\end{align*}
où $h_{ii} = \frac{1}{n} + \frac{(\bar{x} - x_i)^2}{S_{XX}}$.



\section{Régression linéaire multiple}
% \mathrm{•} utilise la notation matricielle
% \bm{•} est utilisé pour la notation matricielle, mais lorsqu'il y a des symboles mathématiques.
\subsection*{Le modèle et ses propriétés}
\begin{align*}
\matr{Y}_{n \times 1} & = \matr{X}_{n \times p'} \bm{\beta}_{p' \times 1} + \bm{\varepsilon}_{n \times 1} \\
\esp{\matr{Y}}	& = \matr{X} \bm{\beta} \quad , Var(\matr{Y}) = \sigma^2 \matr{I}_{n \times n} \\
Y & \overset{H_4}{\sim} N_n(\matr{X} \bm{\beta}, \sigma^2 \matr{I}_{n \times n}) \\
\end{align*}

\subsection*{Paramètres du modèle}
\subsubsection*{Estimation et propriétés des paramètres}
\begin{align*}
\hat{\bm{\beta}} & = \matr{(X^\top X)^{-1} X^\top Y} \\
\esp{\hat{\bm{\beta}}}	& = \bm{\beta} \quad , Var(\matr{Y}) = \sigma^2 (\matr{X^\top X)^{-1}} \\
\hat{\bm{\beta}} & \overset{H_4}{\sim} N_p(\bm{\beta}, \sigma^2 (\matr{X^\top X)^{-1}})
\end{align*}
\subsubsection*{Intervalle de confiance sur les paramètres}
\begin{align*}
\left[ \hat{\bm{\beta}} \pm t_{n-p'}(1- \frac{\alpha}{2}) \sqrt{s^2 v_{jj}} \right]
\end{align*}
où $v_{jj}$ est l'élément $(i,i)$ de la matrice $\matr{(X^\top X)^{-1}}$.
\subsubsection*{Estimation de $\sigma^2$}
\begin{align*}
\hat{\sigma}^2 = s^2 = \frac{\hat{\bm{\varepsilon}}^\top \hat{\bm{\varepsilon}}}{n-p'}
\end{align*}

\subsubsection*{Test d'hypothèse sur un paramètre du modèle}
On rejète $H_0 : \beta_j = 0$ si
\begin{align*}
|t_{obs, j}| = \frac{\beta_j \sqrt{n-p'}}{\sqrt{v_{jj} \hat{\bm{\varepsilon}}^\top \hat{\bm{\varepsilon}}}} > t_{n-p'}\left(1 - \frac{\alpha}{2} \right) \\
\end{align*}

\subsection*{Propriétés de la droite de régression}
\begin{align*}
\hat{\matr{Y}}	& = \matr{X} \bm{\beta} \\
	& = \matr{X(X^\top X)^{-1} X^\top Y} \\
	& = \matr{H Y} 
\end{align*}
où $\matr{H = X(X^\top X)^{-1}X^\top}$ est la \textit{hat matrix}.

On a aussi que
\begin{align*}
\esp{\hat{\matr{Y}}} & = \matr{X} \bm{\beta} \quad , Var(\hat{\matr{Y}}) = \sigma^2 \matr{H} \\
\hat{\matr{Y}} & \overset{H_4}{\sim} N_n(\matr{X} \bm{\beta} , \sigma^2 \matr{H}) \\
\end{align*}

Pour les résidus de la droite de régression, on a
\begin{align*}
\esp{\hat{\bm{\varepsilon}}}  \overset{H_1}{=} 0 \quad , Var(\hat{\bm{\varepsilon}}) = \sigma^2(\matr{I}_{n \times n} - \matr{H}) \\
\hat{\bm{\varepsilon}} \overset{H_4}{\sim} N_n(0, \sigma^2 (\matr{I}_{n \times n} - \matr{H})) \\
\end{align*}


\subsection*{Intervalle de confiance pour la prévision}
\subsubsection*{Théorème de Gauss-Markov}
Selon les postulats $H_1$ à $H_4$, l'estimateur
\begin{align*}
\mathrm{a^\top} \hat{\bm{\beta}} = \matr{a^\top (X^\top X)^{-1} X^\top Y}
\end{align*}
est le meilleur estimateur pour $\matr{a^\top} \bm{\beta}$

(BLUE : \textit{Best linear unbiaised estimator}).

\subsubsection*{I.C. pour la prévision de la valeur moyenne $\esp{\matr{Y} | \matr{X^*}}$}
\begin{align*}
\left[ \matr{{X^*}^\top} \hat{\bm{\beta}} \pm t_{n-p'}\left(1 - \frac{\alpha}{2} \right) \sqrt{s^2 \matr{{X^*}^\top} (X^\top X)^{-1}{X^*}^\top} \right]
\end{align*}

\subsubsection*{I.C. pour la valeur prédite $\hat{\matr{Y}}|\matr{X^*}$}
\begin{align*}
\left[ \matr{{X^*}^\top} \hat{\bm{\beta}} \pm t_{n-p'}\left(1 - \frac{\alpha}{2} \right) \sqrt{s^2\left( 1 +  \matr{{X^*}^\top} (X^\top X)^{-1}{X^*}^\top\right)} \right]
\end{align*}


\subsection*{Analyse de la variance}
\subsubsection*{Tableau ANOVA}
\begin{itemize}
\item On utilise le même tableau ANOVA qu'en régression linéaire simple.
\item $SSR_{\text{régression}} = \sum_{i=1}^{p} SSR_i $, où $SSR_i$ représente le SSR individuel de la variable explicative $i$ calculé par R. On peut ensuite trouver $MSR$ et la statistique $F_{obs}$.
\end{itemize}

\subsubsection*{Test F pour la validité globale de la régression}
Même test qu'en régression linéaire simple.


\subsubsection*{Test F partiel pour la réduction du modèle}
Avec $k < p$, on va rejeter
\begin{align*}
H_0 : Y_i = \beta_0 + \beta_1 x_{i1} + ... \beta_{ik} \quad \text{(modèle réduit)}
\end{align*}
Pour
\begin{align*}
H_1 : Y_i = \beta_0 + \beta_1 x_{i1} + ... \beta_{ip} \quad \text{(modèle complet)}
\end{align*}
Si
\begin{align*}
F_{obs} = \frac{(SSE^{(0)} - SSE^{(1)}) / \Delta dl}{SSE^{(1)} / (n-p')} \geq F_{p-k, n-p'}(1- \alpha)
\end{align*}
où $\Delta dl = p - k$, $SSE^{(0)}$ pour le modèle réduit ($H_0$) et $SSE^{(1)}$ pour le modèle complet ($H_1$).






\end{multicols*}
%% -----------------------------
%% Fin du document
%% -----------------------------
\end{document}