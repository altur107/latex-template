% Template entête TP
% Université Laval

% ATTENTION : la référence à ce document (\input) doit être fait après \begin{document} et l'appel aux fonctions \author, \title et \date dans le préambule.

% Copier-coller ce script dans le même dossier que le fichier .tex principal, puis insérer la commande suivante à l'endroit où l'on veut voir apparaître la page titre : 
% \input{pagetitre_UL}

%%%%%%%%%%%%%%%%%%%%%%%%%%%%%%
\makeatletter
\begin{titlepage}
\centering \large

% ----- COÉQUIPIER 1 -----
\@author
\par
$(111 \ 126 \ 819)$
\vspace{0.5cm}
%-----

% ----- COÉQUIPIER 2 -----
% Nom et prénom
% \par
% $(000 \ 000 \ 000)$
% \vspace{0.5cm}
%-----

% ----- COÉQUIPIER 3 -----
% Nom et prénom
% \par
% $(000 \ 000 \ 000)$
% \vspace{0.5cm}
%-----

% ----- COÉQUIPIER 4 -----
% Nom et prénom
% \par
% $(000 \ 000 \ 000)$
% \vspace{0.5cm}
%-----

%------------------------------
\vfill


% ----- NOM DU COURS -----
{
\scshape
Introduction à l'actuariat II
\par
ACT-2001
}
%------------------------------
\vfill


% ----- TITRE DU TRAVAIL -----
{
\bfseries \Large
\@title
}
%------------------------------
\vfill


% ----- NOM DU PROFESSEUR -----
Travail présenté à
\par
Étienne Marceau
%------------------------------
\vfill

% ----- BAS DE PAGE -----
École d'actuariat
\par
Université Laval
\par 
\@date
%------------------------------
\end{titlepage}
\makeatother
%%%%%%%%%%%%%%%%%%%%%%%%%%%%%%